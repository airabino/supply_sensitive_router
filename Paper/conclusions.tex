\section*{Conclusions}

Personal vehicles play a major role in the transportation systems of all major developed economies globally. The role of personal vehicles is particularly pronounced in the US. As the transition to \glspl{bev} continues, \glspl{bev} will be pressed into more unsuitable roles and less forgiving drivers. Where \glspl{bev} provide advantages for routing travel where long dwells enable slow charging events to suffice, they are disadvantageous for long-trip travel. As many customers disproportionately weight long trips in purchase decisions, public money has been poured into subsidy programs to build out a national DC charging network in the US. This paper provides a framework for the quantitative analysis of the performance of said network from the point of view of transportation accessibility. The framework proposed accounts for the effects of vehicle and behavioral parameters as well as the precise structure of the network. The current DC charging network in California is analyzed and shown to enable \gls{bev} travel reasonably well. Outcomes are found to be also dependent on vehicle range and maximum charge rate and individual risk attitude. If vehicles can charge at existing Tesla stations then the time difference between \gls{bev} and \gls{icev} is found to be mostly due to unavoidable differences in range-addition rates. Where Tesla chargers are not accessible, drivers can expect substantial additional time due to route deviations and queuing. The structure of the Tesla network with relatively few stations and relatively many ports per station is particularly conducive to long-trip travel. Current market dynamics have pushed the non-Tesla networks towards the opposite structure. Public investment may be more productively used to encourage more Tesla-like networks. However, even with a substantially improved DC charging network, \gls{bev} travel will not achieve total-time parity with \glspl{icev} unless charging speeds are massively increased. thus investments in enabling inter-city vehicle travel should be considered alongside investments inn other low-carbon inter-city travel modes.